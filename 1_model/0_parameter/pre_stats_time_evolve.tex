\documentclass[tikz]{standalone}
\usepackage{float}              % [H]
\usepackage{graphicx}           % (pdf, png, jpg, eps)
\usepackage{pgfplots}
\usepackage{siunitx}
\usepackage{currfile}
\usepackage{ifthen}
\usepackage{tikz}
\usetikzlibrary{calc}
\usepackage{pgfplots}
\usepackage{pgfplotstable}
\pgfplotsset{compat=1.17}
\usepgfplotslibrary{colorbrewer}
\usepgfplotslibrary{statistics}
\usepgfplotslibrary{groupplots}
\tikzset{>=latex}
% 
% \usetikzlibrary{external}
% \tikzexternalize[prefix=./output/tmp/]
% 
\def\width{13.87303cm}
\def\height{22.37076cm}
\def\datapath{../../__PATH__/time_evolve}
\directlua0{
  pdf.setinfo ("/Path: (__PATH__)")
}
% 
\def\vrf{\langle{r_f}\rangle}
\def\vfl{\nu_{l}}
\def\vfr{\nu_{r}}
% 
\begin{document}
% 
\pgfplotsset{every axis plot post/.append style={%
   only marks, mark=*,
   mark size=0.3pt, %mean value marker
   error bars/.cd,
   error bar style={line width=0.4pt}, % thickness line top to bottom
   error mark options={rotate=90,mark size=0.8pt, line width=0.4pt}, % top and bottom (line) marker
   y dir=both, y explicit,
   },
   ylabel absolute, every axis y label/.append style={yshift=0.0ex},
   }
%
\pgfplotsset{colormap/Blues-4}
\pgfplotsset{colormap/Greens-4}
\pgfplotsset{colormap/RdPu-4}
\pgfplotsset{colormap/Purples-4}
\pgfplotsset{colormap/Reds-4}
\pgfplotsset{colormap/Greys-4}
% 
\pgfplotsset{
cycle multiindex* list={
   index of colormap={1 of Blues-4},index of colormap={2 of Blues-4},
   index of colormap={3 of Blues-4},
   index of colormap={1 of Greens-4},index of colormap={2 of Greens-4},
   index of colormap={3 of Greens-4},
   index of colormap={1 of RdPu-4},index of colormap={2 of RdPu-4},
   index of colormap={3 of RdPu-4},
   index of colormap={1 of Purples-4},index of colormap={2 of Purples-4},
   index of colormap={3 of Purples-4},
   index of colormap={1 of Reds-4},index of colormap={2 of Reds-4},
   index of colormap={3 of Reds-4},
   index of colormap={1 of Greys-4},index of colormap={2 of Greys-4},
   index of colormap={3 of Greys-4},
   \nextlist
}}
% 
\newcommand{\Loop}[1]{
   \foreach \r in {0.5,1.0,2.0}{
   \foreach \fl in {2.0, 8.0}{
   \foreach \fr in {2.0,4.0,8.0}{#1}}}} % 1.0 und 2.0 nicht unterscheidbar
% 
\foreach \psi in {0.5,1.0}{
\begin{tikzpicture}
\begin{groupplot}[%
  group style={
      group size=1 by 3,
      horizontal sep=2cm,
      vertical sep=0.75cm,
      ylabels at=edge left,
  },
   width = \width,
   height = 0.275*\height,
   xmode=log,
   ymode=log,
]
% 
% \nextgroupplot[ylabel={overlap}]
   % \Loop{
%     \addplot+[] table[
%         x={times_mean}, y={overlaps_mean}, x error = {times_std} , y error={overlaps_std}, col sep=comma,]
%         {\datapath/cube_stat_time_evolve_r_\r_psi_0.5_fr_\fr_fl_\fl_.csv};
%    }
% % 
% \nextgroupplot[ylabel={$\# obj$}]
   % \Loop{
%     \addplot+[] table[
%         x={times_mean}, y={num_objs_mean}, x error = {times_std} , y error={num_objs_std}, col sep=comma,]
%         {\datapath/cube_stat_time_evolve_r_\r_psi_0.5_fr_\fr_fl_\fl_.csv};
%    }
% 
\nextgroupplot[ylabel={$\# steps$}, ymin=30, ymax=200000, xmin=0.2, xmax=100000]
   \Loop{
    \addplot+[] table[
        x={times_mean}, y={steps_mean}, x error = {times_std} , y error={steps_std}, col sep=comma,]
        {\datapath/cube_stat_time_evolve_r_\r_psi_\psi_fr_\fr_fl_\fl_.csv};
   }
   \draw[->] (rel axis cs:0.9,0.05) -- (rel axis cs:0.975,0.05) node[midway, anchor=south]{$t/s$};
% 
\nextgroupplot[ylabel={$\# col obj$}, ymin=0, ymax=2000000, xmin=0.2, xmax=100000]
   \Loop{
    \addplot+[] table[
        x={times_mean}, y={num_col_objs_mean}, x error = {times_std} , y error={num_col_objs_std}, col sep=comma,]
        {\datapath/cube_stat_time_evolve_r_\r_psi_\psi_fr_\fr_fl_\fl_.csv};
   }
   \draw[->] (rel axis cs:0.9,0.05) -- (rel axis cs:0.975,0.05) node[midway, anchor=south]{$t/s$};
%
\nextgroupplot[ylabel={$overlap frac$}, ymin=0.0005, ymax=0.3, xmin=0.2, xmax=100000]
    \Loop{
    \addplot+[] table[
        x={times_mean}, y={overlaps_frac_mean}, x error = {times_std} , y error={overlaps_frac_std}, col sep=comma,]
        {\datapath/cube_stat_time_evolve_r_\r_psi_\psi_fr_\fr_fl_\fl_.csv};
    }
    \draw[->] (rel axis cs:0.9,0.05) -- (rel axis cs:0.975,0.05) node[midway, anchor=south]{$t/s$};
   % \coordinate (L) at (axis description cs:1,0);
% 
% 
\end{groupplot}
% 
% LEGEND
\begin{pgfinterruptboundingbox}
\begin{axis}[%
   legend to name={legend\currfilebase},
   legend cell align=left,
   legend columns=6,
   legend transposed=true,
   scale only axis, 
   width=1mm,
   hide axis,
   legend image post style={sharp plot},
   legend style={font=\vphantom{hg},draw=white!15!black,legend cell align=left, 
                 /tikz/every even column/.append style={column sep=2ex},
                 nodes={scale=0.85, transform shape}},
   ]
   % 
   % \pgfmathtruncatemacro\counter{0}
   \Loop{
         \addplot coordinates {(0,0)};
         \pgfmathparse{int(round(\fl))} \let\intfl=\pgfmathresult
         \pgfmathparse{int(round(\fr))} \let\intfr=\pgfmathresult
         \ifthenelse{\intfr = 4}{%
            \addlegendentryexpanded{$\vfr=\intfr,\vfl=\intfl,\vrf=\r$}}{%
            \addlegendentryexpanded{$\vfr=\intfr$}}
% 
         % \pgfmathtruncatemacro\counter{\counter+1}
         % \ifthenelse{\counter = 3}{\addlegendimage{empty legend}\addlegendentry{}}{}
         % \ifthenelse{\counter = 6}{\addlegendimage{empty legend}\addlegendentry{}}{}
         % \ifthenelse{\counter = 12}{\addlegendimage{empty legend}\addlegendentry{}}{}
         % \ifthenelse{\counter = 15}{\addlegendimage{empty legend}\addlegendentry{}}{}
   }
\end{axis}
\end{pgfinterruptboundingbox}
% \path node[anchor=south] at ($(current bounding box.north) + (0,1)$) {\ref{legend\currfilebase}};
% \path let \p1=($(current bounding box.north east)-(current bounding box.south west)$) in node[anchor=center] at ($(current bounding box.south west) + (\x1/5*4,\y1*1/4)$) {\ref{legend\currfilebase}};
% \path node[anchor=north] at ($(current bounding box.south) - (0,0.5cm)$) {\ref{legend\currfilebase}};
\path node[anchor=north] at ($(group c1r3.south) - (0,0.75cm)$) {\ref{legend\currfilebase}};
\end{tikzpicture}
}
\end{document}
