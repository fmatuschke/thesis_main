\documentclass[tikz]{standalone}
\usepackage{float}              % [H]
\usepackage{graphicx}           % (pdf, png, jpg, eps)
\usepackage{pgfplots}
\usepackage{siunitx}
\usepackage{currfile}
\usepackage{ifthen}
\usepackage{tikz}
\usetikzlibrary{calc}
\usepackage{pgfplots}
\usepackage{pgfplotstable}
\pgfplotsset{compat=1.17}
\usepgfplotslibrary{colorbrewer}
\usepgfplotslibrary{polar}
\usepgfplotslibrary{groupplots}
\usepgfplotslibrary{statistics}
\tikzset{>=latex}
% 
\newlength{\width}
\newlength{\height}
\setlength{\width}{13.87303cm} %5.462 inch
\setlength{\height}{22.37076cm} %8,807 inch
\def\datapath{../../__PATH__/hist}
\directlua0{
  pdf.setinfo ("/Path: (__PATH__)")
}
% 
\def\vrf{\langle{r_f}\rangle}
\def\vfl{\nu_{l}}
\def\vfr{\nu_{r}}
% 
\definecolor{dark21}{HTML}{1b9e77}
\definecolor{dark22}{HTML}{d95f02}
\definecolor{dark23}{HTML}{7570b3}
\colorlet{RED}{dark22}
\colorlet{GREEN}{dark21}
\colorlet{BLUE}{dark23} 
% 
\begin{document}
% 
\pgfplotsset{%
  colormap/cividis/.style={colormap={cividis}{[1pt]
    rgb(0pt)=(0.00000000, 0.13511200, 0.30475100);
    rgb(25pt)=(0.25974000, 0.30512000, 0.42281000);
    rgb(50pt)=(0.48514100, 0.48245100, 0.47038400);
    rgb(75pt)=(0.73242200, 0.67736400, 0.42571700);
    rgb(100pt)=(0.99573700, 0.90934400, 0.21777200);
}}}
% 
\pgfplotsset{%
   PStyle/.style={%
      every axis/.append style={font=\small},
      width = 0.17\width, height = 0.17\width,
      scale only axis,
      xtick={0,45,...,315},
      ytick={60,30},
      yticklabels={$\SI{30}{\degree}$,$\SI{60}{\degree}$},
      yticklabel style=white,
      tickwidth=0, xtick distance = 45,
      separate axis lines, y axis line style= { draw opacity=0 },
      ymin=0, ymax=90,
      axis on top=true,
      colormap/cividis,
      % colorbar, colorbar style={ytick=\empty, width=5pt}, 
   }
}
% 
\begin{tikzpicture}[]
% 
\def\p{3}
\def\delta{3.3}
\foreach \o[count=\oi from 0] in {0,30,60,90}{
   %
   \pgfmathparse{2*1.05*\delta*Mod(\oi,2)} \pgfmathsetmacro{\px}{\pgfmathresult}
   \pgfmathparse{1*1.05*\delta*div(\oi,2)} \pgfmathsetmacro{\py}{\pgfmathresult}
   % 
   \begin{scope}[shift={(\px + 0.00,-\py)}]
      \begin{polaraxis}[
         PStyle,
      ]
         % FIXME: model, TODO: for now only dummy
         \addplot [matrix plot*,point meta=\thisrowno{2},] table[] {%
            \datapath/sim_hists_p_0.\p_o_\o.0_r_0.5_f0_0.0_f1_90.0.dat};
      \end{polaraxis}
   \end{scope}
   %
   % simulation
   \begin{scope}[shift={(\px + \delta,-\py)}]
      \begin{polaraxis}[
         PStyle,
      ]
         \addplot [matrix plot*,point meta=\thisrowno{2},] table[] {%
            \datapath/sim_hists_p_0.\p_o_\o.0_r_0.5_f0_0.0_f1_90.0.dat};
      \end{polaraxis}
   \end{scope}
}
% 
% % center axis
% \coordinate (PA) at (group c1r2.west);
% \coordinate (PB) at (group c2r2.east);
% \coordinate (PC) at (current bounding box.west);
% \path[] let \p1 = ($(PA)-(PC)$) in ($(PA) - (\x1,0)$) -- ($(PB) + (\x1,0)$);
% % 
\end{tikzpicture}
% 
\begin{tikzpicture}[]
% 
\def\p{5}
\def\delta{3.3}
\foreach \o[count=\oi from 0] in {0,30,60,90}{
   %
   \pgfmathparse{2*1.05*\delta*Mod(\oi,2)} \pgfmathsetmacro{\px}{\pgfmathresult}
   \pgfmathparse{1*1.05*\delta*div(\oi,2)} \pgfmathsetmacro{\py}{\pgfmathresult}
   % 
   \begin{scope}[shift={(\px + 0.00,-\py)}]
      \begin{polaraxis}[
         PStyle,
      ]
         % FIXME: model, TODO: for now only dummy
         \addplot [matrix plot*,point meta=\thisrowno{2},] table[] {%
            \datapath/sim_hists_p_0.\p_o_\o.0_r_0.5_f0_0.0_f1_90.0.dat};
      \end{polaraxis}
   \end{scope}
   %
   % simulation
   \begin{scope}[shift={(\px + \delta,-\py)}]
      \begin{polaraxis}[
         PStyle,
      ]
         \addplot [matrix plot*,point meta=\thisrowno{2},] table[] {%
            \datapath/sim_hists_p_0.\p_o_\o.0_r_0.5_f0_0.0_f1_90.0.dat};
      \end{polaraxis}
   \end{scope}
}
% 
% % center axis
% \coordinate (PA) at (group c1r2.west);
% \coordinate (PB) at (group c2r2.east);
% \coordinate (PC) at (current bounding box.west);
% \path[] let \p1 = ($(PA)-(PC)$) in ($(PA) - (\x1,0)$) -- ($(PB) + (\x1,0)$);
% % 
\end{tikzpicture}
% 
\end{document}
