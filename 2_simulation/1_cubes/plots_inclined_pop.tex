\documentclass[tikz]{standalone}
\usepackage{float}              % [H]
\usepackage{graphicx}           % (pdf, png, jpg, eps)
\usepackage{pgfplots}
\usepackage{siunitx}
\usepackage{currfile}
\usepackage{ifthen}
\usepackage{tikz}
\usetikzlibrary{calc}
\usepackage{pgfplots}
\usepackage{pgfplotstable}
\pgfplotsset{compat=1.17}
\usepgfplotslibrary{colorbrewer}
\usepgfplotslibrary{groupplots}
\usepgfplotslibrary{statistics}
\usepgfplotslibrary{fillbetween}
\tikzset{>=latex}
\DeclareSIUnit{\arbitraryunit}{a.u.}
% 
\newlength{\width}
\newlength{\height}
\setlength{\width}{13.87303cm} %5.462 inch
\setlength{\height}{22.37076cm} %8,807 inch
\def\datapath{../../__PATH__/analysis}
\directlua0{
  pdf.setinfo ("/Path: (__PATH__)")
}
% 
../../thesis/content/parameters.tex
% 
\definecolor{dark21}{HTML}{1b9e77}
\definecolor{dark22}{HTML}{d95f02}
\definecolor{dark23}{HTML}{7570b3}
\colorlet{RED}{dark22}
\colorlet{GREEN}{dark21}
\colorlet{BLUE}{dark23} 
% 
\pgfplotsset{
   every tick label/.append style={font=\small,},
   title style={font=\normalsize,},
   Boxplot/.style={
      mark=*,
      mark size=0.21pt,
      mark options={fill=black},
      boxplot={%
         draw position={\plotnumofactualtype},
         draw direction=y,
         draw/median/.code={%
            \draw[black,/pgfplots/boxplot/every median/.try]
               (boxplot box cs:\pgfplotsboxplotvalue{median},0) --
               (boxplot box cs:\pgfplotsboxplotvalue{median},1);
            % add circle for visibility
            \draw[mark size=1.5pt,/pgfplots/boxplot/every median/.try]
            \pgfextra
            \pgftransformshift{
               \pgfplotsboxplotpointabbox
                  {\pgfplotsboxplotvalue{median}}{0.5}
            }
            \pgfuseplotmark{*}
            \endpgfextra
            ;
         },
      },
      fill=GREEN, draw=black, solid,
   },
}
% 
\newcommand{\plotgt}[3]{
\pgfplotsset{update limits=false}
\def\nboxes{9}
\addplot+[draw=none, mark=none, name path=A, smooth] table[%
   x expr={\thisrow{#2}/90*\nboxes}, y={#3_25}, col sep=comma, row sep=newline,
   ]{#1};
% 
\addplot+[mark=none, BLUE!50!black, smooth, densely dashdotted] table[%
   x expr={\thisrow{#2}/90*\nboxes}, y={#3_50}, col sep=comma, row sep=newline,
   ]{#1};
\addplot+[draw=none, mark=none, name path=B, smooth] table[%
   x expr={\thisrow{#2}/90*\nboxes}, y={#3_75}, col sep=comma, row sep=newline,
   ]{#1};
\addplot[BLUE!75, smooth] fill between [of=A and B];
\pgfplotsset{cycle list shift=-7}
\pgfplotsset{update limits=true}
}
%  
\begin{document}
% 
% \foreach \psi in {0.7} {
\foreach \psi in {0.1,0.2,0.3,0.4,0.5,0.6,0.7,0.8,0.9} {
% 
\directlua{tex.print('\string\\message{\psi}')}
% 
\begin{tikzpicture}[]
% 
\begin{groupplot}[%
   group style={
      group size=2 by 4,
      horizontal sep=1cm,
      vertical sep=1cm,
      ylabels at=edge left,
   },
   width = 0.495\width,
   height = 0.21\height,
   xmin=-0.5, xmax=9.5,
   xtick={0,3,6,9},
   xticklabels={$\SI{0}{\degree}$, $\SI{30}{\degree}$, $\SI{60}{\degree}$, $\SI{90}{\degree}$},
   xlabel={$\modelOmega$},
   every axis x label/.style={
      at={(ticklabel* cs:1.0,0.0)},
      anchor=north west,
   },
   every axis title/.append style={
      yshift=-8pt,
   },
   enlarge y limits,
   ymajorgrids,
   clip=true,
]
% 
\nextgroupplot[title={\vphantom{/}transmittance}]
\foreach \omega in {0,10,20,30,40,50,60,70,80,90} {      
   \addplot+[Boxplot] table[%
      y={epa_trans}, col sep=comma, row sep=newline,
      ]{\datapath/cube_2pop_135_rc1_inclined_plots_inclined_pop_psi_\psi_omega_\omega.csv};
}
% 
% 
\nextgroupplot[title={\vphantom{/}retardation}]
\foreach \omega in {0,10,20,30,40,50,60,70,80,90} {      
   \addplot+[Boxplot] table[%
      y={epa_ret}, col sep=comma, row sep=newline,
      ]{\datapath/cube_2pop_135_rc1_inclined_plots_inclined_pop_psi_\psi_omega_\omega.csv};
}
% 
% 
\nextgroupplot[title={\vphantom{/}direction / $\si{\degree}$}]
\plotgt{\datapath/cube_2pop_135_rc1_inclined_plots_inclined_pop_psi_\psi_model.csv}{omega}{phi}
\foreach \omega in {0,10,20,30,40,50,60,70,80,90} {      
   \addplot+[Boxplot] table[%
      y={rofl_dir}, col sep=comma, row sep=newline,
      ]{\datapath/cube_2pop_135_rc1_inclined_plots_inclined_pop_psi_\psi_omega_\omega.csv};
}
\pgfplotsset{update limits=false}
\addplot+[mark=none, black, densely dotted]  coordinates {(0,0) (9,0)};
\pgfplotsset{update limits=true}
% 
% 
\nextgroupplot[title={\vphantom{/}inclination / $\si{\degree}$}]
\plotgt{\datapath/cube_2pop_135_rc1_inclined_plots_inclined_pop_psi_\psi_model.csv}{omega}{alpha}
\foreach \omega in {0,10,20,30,40,50,60,70,80,90} {      
   \addplot+[Boxplot] table[%
      y={rofl_inc}, col sep=comma, row sep=newline,
      ]{\datapath/cube_2pop_135_rc1_inclined_plots_inclined_pop_psi_\psi_omega_\omega.csv};
}
\pgfplotsset{update limits=false}
% individual populations
\addplot+[mark=none, black, densely dotted]  coordinates {(0,0) (9,0)};
\addplot+[mark=none, black, densely dotted]  coordinates {(0,0) (9,90)};
% circmean
\addplot+[mark=none, white, smooth,
   dash=on 3.4pt off 2.6pt phase 0.2pt, line width=0.8pt]  table[%
   x={x}, y={y}, col sep=comma, row sep=newline,
   ]{\datapath/cube_2pop_135_rc1_inclined_plots_inclined_pop_psi_\psi_theo_incl.csv};
\addplot+[mark=none, black, smooth,
   dash=on 3pt off 3pt phase 0pt, line width=0.4pt]  table[%
   x={x}, y={y}, col sep=comma, row sep=newline,
   ]{\datapath/cube_2pop_135_rc1_inclined_plots_inclined_pop_psi_\psi_theo_incl.csv};
\pgfplotsset{update limits=true}
% 
% 
\nextgroupplot[title={\vphantom{/}trel}]
\foreach \omega in {0,10,20,30,40,50,60,70,80,90} {      
   \addplot+[Boxplot] table[%
      y={rofl_trel}, col sep=comma, row sep=newline,
      ]{\datapath/cube_2pop_135_rc1_inclined_plots_inclined_pop_psi_\psi_omega_\omega.csv};
}
% 
% 
\nextgroupplot[title={\vphantom{/}$\openingAngle$ / $\si{\degree}$}]
\plotgt{\datapath/cube_2pop_135_rc1_inclined_plots_inclined_pop_psi_\psi_model.csv}{omega}{domega}
\foreach \omega in {0,10,20,30,40,50,60,70,80,90} {      
   \addplot+[Boxplot] table[%
      y={domega}, col sep=comma, row sep=newline,
      ]{\datapath/cube_2pop_135_rc1_inclined_plots_inclined_pop_psi_\psi_omega_\omega.csv};
}
% 
% 
\nextgroupplot[title={\vphantom{/}acc},
   xmin=-5, xmax=95,
   xtick={0,30,60,90},]
   \addplot+[mark options={fill=GREEN,scale=0.75},GREEN]
      table[x={omega},y={acc},col sep=comma, row sep=newline,]
      {\datapath/cube_2pop_135_rc1_inclined_plots_inclined_pop_schilling_psi_\psi.csv};
% 
% 
\nextgroupplot[title={\vphantom{/}R}]
\foreach \omega in {0,10,20,30,40,50,60,70,80,90} {      
   \addplot+[Boxplot] table[%
      y={R}, col sep=comma, row sep=newline,
      ]{\datapath/cube_2pop_135_rc1_inclined_plots_inclined_pop_psi_\psi_omega_\omega.csv};
}
\end{groupplot}
% 
% LEGENDE
\begin{axis}[%
   hide axis, scale only axis,
   xmin=10,xmax=50,ymin=0,ymax=0.4, % dummy
   legend to name={legend1},
   legend style={draw=white!15!black,legend cell align=left},
   legend columns=4,
   legend style={scale=0.5,/tikz/every even column/.append style={column sep=1ex}},
   ]
   \addlegendimage{black,mark=none, densely dotted}
   \addlegendentry{pop orientation};
   % 
   \addlegendimage{black,mark=none, dash=on 3pt off 3pt phase 0pt, line width=0.4pt}
   \addlegendentry{theoretical};
   % 
   \addlegendimage{mark=none, BLUE!50!black, smooth, densely dashdotted}
   \addlegendentry{model median};
   % 
   \addlegendimage{area legend,BLUE!75,fill=BLUE!75,only marks}
   \addlegendentry{model $\SI{25}{\percent}-\SI{75}{\percent}$};
\end{axis}
% 
% \coordinate (CA) at (group c1r1.north);
% \coordinate (CB) at (group c2r1.north);
\pgfmathparse{int(round(\psi*10))*10}
\node[anchor=south] at ($(group c1r1.north)!0.5!(group c2r1.north) + (0,0.5)$) {inclined fiber pop., $\modelPsi = \SI{\pgfmathresult}{\percent}$};
\node[anchor=north, scale=0.75] at 
   ($(group c1r4.south)!0.5!(group c2r4.south) - (0,0.5)$) {\ref{legend1}};
% 
% center axis
\coordinate (PA) at (group c1r2.west);
\coordinate (PB) at (group c2r2.east);
\coordinate (PC) at (current bounding box.west);
\path[] let \p1 = ($(PA)-(PC)$) in ($(PA) - (\x1,0)$) -- ($(PB) + (\x1,0)$);
% 
\end{tikzpicture}
}
% 
\end{document}
