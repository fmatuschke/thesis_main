\documentclass[tikz]{standalone}
% 
\usepackage{float}              % [H]
\usepackage{graphicx}           % (pdf, png, jpg, eps)
\usepackage{tikz}
\usepackage{pgfplots}
\usepackage{pgfplotstable}
\usepackage{siunitx}
\usepackage{currfile}
\usepackage{ifthen}
\pgfplotsset{compat=1.17}
\usepgfplotslibrary{polar}
\usetikzlibrary{external}
\usetikzlibrary{calc}
% 
% \tikzexternalize[prefix=./tikz/,
%                %   figure name=\radius_\setup_\species_\model_,
%                 %  mode=list and make, 
%                  ]
% 
\def\datapath{../../output/sim_120_ime}
% 
\def\width{13.87303cm}
\def\height{22.37076cm}
% 
\def\setup{__MICROSCOPE__}
\def\species{__SPECIES__}
\def\model{__MODEL__}
\def\radius{__RADIUS__}
\def\name{__NAME__}
\def\norm{__NORM__}
% 
\begin{document}
% 
% 
\def\delta{4.25}
\pgfplotsset{every tick label/.append style={font=\footnotesize}}
% 
\begin{tikzpicture}
% 
\begin{scope}[local bounding box=bb]
% 
\foreach [count=\i from 0] \fnull in {0.0,30.0,60.0,90.0} {
\foreach [count=\j from 0] \psi in {0.3, 0.6, 0.9} { % 0.0, unecesarry, because other psi at omega==0 (should) have the same value
% 
\begin{scope}[shift={(\j*\delta,-\i*\delta)}, local bounding box=\i\j]
\begin{polaraxis}[
    width=3.25cm, height = 3.25cm,
    scale only axis=true,
    xtick={0,45,...,315},
    ytick={60,30},
    xlabel style={overlay},
    xticklabel style={overlay},
    yticklabels={30,60},
    ylabel style={overlay},
    yticklabel style={white, overlay},
    colormap/viridis,
    point meta min = 0,
    point meta max = 1,
    tickwidth=0,
    xtick distance = 45,
    separate axis lines,
    y axis line style= { draw opacity=0 },
    ymin=0, ymax=90,
    axis on top=true,
]
\addplot [surf,point meta=\thisrowno{2},] table {%
  \datapath/hist/sim_r_\radius_setup_\setup_s_\species_m_\model_psi_\psi_f0_\fnull_\name_\norm_hist.dat}; %, contour filled={number=9}
\addplot [only marks, scatter, scatter src=explicit] file[] {%
  \datapath/hist/sim_r_\radius_setup_\setup_s_\species_m_\model_psi_\psi_f0_\fnull_\name_\norm_data.dat}; %, contour filled={number=9}
\addplot [ only marks, mark=o, thin, dash pattern=on 1pt off 1pt] file[] {%
  \datapath/hist/sim_r_\radius_setup_\setup_s_\species_m_\model_psi_\psi_f0_\fnull_\name_\norm_data.dat}; %, contour filled={number=9}
\addplot [only marks, mark=o, thick] file[] {%
  \datapath/hist/sim_r_\radius_setup_\setup_s_\species_m_\model_psi_\psi_f0_\fnull_\name_\norm_init.dat}; %, contour filled={number=9}
\end{polaraxis}
% 
\end{scope}
\begin{pgfinterruptboundingbox}
\ifthenelse{\i=0}{\node[anchor=south] at ($(\i\j.north)+(0,0.42)$) {$\Psi = \psi$};}{}
\ifthenelse{\j=0}{\node[anchor=south, rotate=90] at ($(\i\j.west)-(0.55,0)$) {$inc = \fnull$};}{}
% \draw (\i\j.north west) rectangle (\i\j.south east);
\end{pgfinterruptboundingbox}
% 
}}
% 
% \draw ($(current bounding box.north west)$) rectangle ($(current bounding box.south east)$);
\path ($(current bounding box.north west)+(-1,1)$) rectangle ($(current bounding box.south east)+(1,-0.42)$);
% 
\end{scope}
% 
% \begin{pgfinterruptboundingbox}
% \draw[very thick, blue, dashed] (bb.north west) rectangle ++ (\width, -\height);
% \end{pgfinterruptboundingbox}
% 
\end{tikzpicture}
% 
\end{document}
